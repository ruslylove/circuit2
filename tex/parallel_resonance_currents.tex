\documentclass[border=10pt]{standalone}
\usepackage{tikz}
\usepackage[american]{circuitikz}
\usepackage{amsmath}

\begin{document}
\begin{circuitikz}[thick]
    % Current Source I
    \draw (0,0) to[isource, l=$I$] (0,4);
    
    % Top wire with node for I (Total Current)
    \draw (0,4) -- (3,4) node[midway, above] {$I$}; 
    \draw[->] (0.5, 4) -- (1.5, 4);

    % Resistor Branch
    \draw (3,4) to[R, l=$R$, v=$V$, i=$I_R$] (3,0);
    
    % Wire to LC tank
    \draw (3,4) -- (6,4) node[midway, above] {$I_{LC}=0$};
    \draw (3,4) -- (6,4); 
    
    % Inductor Branch
    \draw (6,4) to[L, l=$L$, i=$I_L$] (6,0);
    
    % Capacitor Branch
    \draw (6,4) -- (9,4);
    \draw (9,4) to[C, l=$C$, i=$I_C$] (9,0);
    
    % Bottom wire
    \draw (0,0) -- (9,0);
    
    % Ground (optional, but good practice)
    \draw (4.5,0) node[ground]{};
    
    % Arrow for I_LC
    \draw[->] (4, 4) -- (5, 4);

    % Junction dots (bubbles)
    \node[circ] at (3,4) {};
    \node[circ] at (6,4) {};
    \node[circ] at (3,0) {};
    \node[circ] at (6,0) {};

    % Current Loops
    % Loop 1: Source and Resistor
    \draw[->, gray, dashed, rounded corners=5pt] (0.6, 0.6) -- (0.6, 3.4) -- (2.2, 3.4) -- (2.2, 0.6) -- cycle;
    
    % Loop 2: LC Tank (Circulating)
    \draw[->, red, very thick, dashed, rounded corners=5pt] (6.6, 0.6) -- (6.6, 3.4) -- (8.4, 3.4) -- (8.4, 0.6) -- cycle;
    \node[red] at (7.5, 2) {$I_{circ}$};
    
\end{circuitikz}
\end{document}
