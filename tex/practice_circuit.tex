\documentclass[tikz,border=10pt]{standalone}
\usepackage[siunitx]{circuitikz}

\begin{document}
\begin{circuitikz}[american]
    % Circuit 1: Series Vin, R, C, R
    % Loop from (0,0) clockwise
    \draw (0,0) to[sV, l=$V_{in}$] (0,3)
          to[R, l=$1\,\text{k}\Omega$] (3,3)
          to[C, l=$20\,\mu\text{F}$] (6,3)
          to[R, l=$4\,\text{k}\Omega$, v=$V_x$] (6,0)
          to[short] (0,0);
    
    % Circuit 2: Parallel VCCS, C, R
    % Connected at Vin neg (Ground line)
    % "connected at Vin neg and current out of VCCS" is slightly vague, but assuming common ground and separate VCCS stage.
    % Let's shift it to the right.
    \draw (8,0) to[short] (14,0); % Bottom rail
    \draw (6,0) to[short] (8,0); % Connect grounds
    
    % VCCS
    % Assuming VCCS depends on Vx.
    % "parallels of VCCS=Vx/200,C10n,R5k"
    % Let's put VCCS first.
    %\draw (8,0) to[cV, i<=$g_m V_x$, invert] (8,3) circle(0pt); % Just a placeholder, actually use cI
    % Wait, user said VCCS. Controlled Current Source. cI.
    % "current out of VCCS" connected to Vin neg -> Current flows DOWN?
    % Let's draw it flowing DOWN from Top to Bottom rail.
    \draw (8,3) to[cI, l=$\frac{V_x}{200}$] (8,0);
    
    % C10n
    \draw (11,3) to[C, l=$10\,\text{nF}$] (11,0);
    
    % R5k
    \draw (14,3) to[R, l=$5\,\text{k}\Omega$, v=$V_{out}$] (14,0);
    \draw (8,3) -- (14,3); % Top rail
    
    % Ground symbol removed
    
    % Junction bubbles
    \draw (6,0) node[circ]{} (8,0) node[circ]{} (11,0) node[circ]{} (11,3) node[circ]{};
    
\end{circuitikz}
\end{document}
